%% LaTeX-Beamer template for KIT design
%% by Erik Burger, Christian Hammer
%% title picture by Klaus Krogmann
%%
%% version 2.1
%%
%% mostly compatible to KIT corporate design v2.0
%% http://intranet.kit.edu/gestaltungsrichtlinien.php
%%
%% Problems, bugs and comments to
%% burger@kit.edu

\documentclass[18pt]{beamer}
%% SLIDE FORMAT
\usepackage[utf8]{inputenc}
% use 'beamerthemekit' for standard 4:3 ratio
% for widescreen slides (16:9), use 'beamerthemekitwide'

%\usepackage{templates/beamerthemekit}
\usepackage{wrapfig}
\usepackage{hyperref}
\usepackage{algpseudocode}
% \usepackage{templates/beamerthemekitwide}

%\usepackage{epigraph}

% for quotes

\AtBeginSection[] % Do nothing for \section*
{
\begin{frame}<beamer>
\frametitle{Gliederung}
\tableofcontents[currentsection]
\end{frame}
}


\usepackage{biolinum}

%% TikZ INTEGRATION

% use these packages for PCM symbols and UML classes
% \usepackage{templates/tikzkit}
% \usepackage{templates/tikzuml}

% the presentation starts here

\title[Algo I Tut]{Algorithmen Tutorium I}
\subtitle{Pseudocode und mehr}
\author[Zangerle]{Konstantin Zangerle}

\institute{Institut für Theoretische Informatik}

\usepackage{listings}
\usepackage{color}

\definecolor{mygreen}{rgb}{0,0.6,0}
\definecolor{mygray}{rgb}{0.5,0.5,0.5}
\definecolor{mymauve}{rgb}{0.58,0,0.82}

\lstset{ %
  backgroundcolor=\color{white},   % choose the background color
  basicstyle=\footnotesize,        % size of fonts used for the code
  breaklines=true,                 % automatic line breaking only at whitespace
  captionpos=b,                    % sets the caption-position to bottom
  commentstyle=\color{mygreen},    % comment style
  escapeinside={\%*}{*)},          % if you want to add LaTeX within your code
  keywordstyle=\color{blue},       % keyword style
  stringstyle=\color{mymauve},     % string literal style
}

% Bibliography

\begin{document}

% change the following line to "ngerman" for German style date and logos
\selectlanguage{ngerman}

%title page
\begin{frame}
\titlepage
\end{frame}

%table of contents
\begin{frame}{Gliederung}
 \tableofcontents
\end{frame}

\section{Organisatorisches}
\subsection{Tutorium}

\begin{frame}{Tutorium}
Über mich
\begin{itemize}
 \item Konstantin Zangerle
 \item info@konstantinzangerle.de
 \item 4. Semester
 \item Vorher Programmieren Tutor, jetzt Algo
 \item Source auf Github
 \item Kein Ersatz für VL
\end{itemize}

  \begin{quotation}
 Keiner muss ins Tutorium kommen. \\
 Jeder darf aber.  
  \end{quotation}

 
\end{frame}

\subsection{Übungsblätter}
\begin{frame}{Übungsblätter}
 \begin{itemize}
  \item Ausgabe: Mittwoch
  \item Abgabe: Freitags \\
    Im Abgabekasten (nebenan)
  \item Partnerarbeit erlaubt und erwünscht!
  
 \end{itemize}

\end{frame}

\subsection{Klausur}
\begin{frame}{Klausur}
 
\end{frame}


\section{Pseudocode}
\begin{frame}{Pseudocode - Einstieg}
 Ausführung von Oliver Thal
\end{frame}

\section{Binäre Suche}
\begin{frame}
 \begin{itemize}
  \item Wie schnell kann ich in Daten suchen?
  \item Arbeitsauftrag: Entwickle einen Algorithmus, der in einem Array nach einem Element sucht.
 \end{itemize}

\end{frame}

\begin{frame}[fragile]{Binäre Suche}
 \begin{algorithmic}
  \Function{Binary Search}{Sorted Array [1..n] of Number a, Number k}
    \State start = 1
    \State ende = n
    \State middle = 1
    \While{start $\neq$ ende}
      \State middle $\gets\lfloor \frac{start + ende}{2} \rfloor$
      \If {$a[middle] > k$}
	\State start $\gets$ middle
      \ElsIf{$a[middle] < k$}
        \State end $\gets$ middle
      \Else
        \State \Return middle
      \EndIf
    \EndWhile
    \State \Return a[start] == k ? start : -1
  \EndFunction
 \end{algorithmic}
\end{frame}

\end{document}
