%% LaTeX-Beamer template for KIT design
%% by Erik Burger, Christian Hammer
%% title picture by Klaus Krogmann
%%
%% version 2.1
%%
%% mostly compatible to KIT corporate design v2.0
%% http://intranet.kit.edu/gestaltungsrichtlinien.php
%%
%% Problems, bugs and comments to
%% burger@kit.edu

\documentclass[18pt]{beamer}
%% SLIDE FORMAT
\usepackage[utf8]{inputenc}
\usepackage{enumitem}
% use 'beamerthemekit' for standard 4:3 ratio
% for widescreen slides (16:9), use 'beamerthemekitwide'

%\usepackage{templates/beamerthemekit}
\usepackage{wrapfig}
\usepackage{hyperref}
\usepackage[noend]{algpseudocode}
\usepackage{graphicx}
\usepackage{epstopdf}

\epstopdfDeclareGraphicsRule{.gif}{png}{.png}{convert gif:#1 png:\OutputFile}
\AppendGraphicsExtensions{.gif}

 \newcommand{\R}{\mathbb{R}}
 \newcommand{\N}{\mathbb{N}}
 \newcommand{\Oh}{\mathcal{O}}
 \newcommand{\oh}{\mathrm{o}}
 \newcommand{\SP}{\mathrm{SP}}

% \usepackage{templates/beamerthemekitwide}

%\usepackage{epigraph}

% for quotes

\AtBeginSection[] % Do nothing for \section*
{
\begin{frame}<beamer>
\frametitle{Gliederung}
\tableofcontents[currentsection]
\end{frame}
}


\usepackage{biolinum}

\usepackage{pgfplots}

%% TikZ INTEGRATION

% use these packages for PCM symbols and UML classes
% \usepackage{templates/tikzkit}
% \usepackage{templates/tikzuml}

% the presentation starts here

\title[Algo I Tut]{7. Algorithmen Tutorium I}
\subtitle{Übungsklausur - Wiederholen und Üben}
\author[Zangerle]{Konstantin Zangerle}

\institute{Institut für Theoretische Informatik}

\usepackage{listings}
\usepackage{color}

\definecolor{mygreen}{rgb}{0,0.6,0}
\definecolor{mygray}{rgb}{0.5,0.5,0.5}
\definecolor{mymauve}{rgb}{0.58,0,0.82}

\lstset{ %
  backgroundcolor=\color{white},   % choose the background color
  basicstyle=\footnotesize,        % size of fonts used for the code
  breaklines=true,                 % automatic line breaking only at whitespace
  captionpos=b,                    % sets the caption-position to bottom
  commentstyle=\color{mygreen},    % comment style
  escapeinside={\%*}{*)},          % if you want to add LaTeX within your code
  keywordstyle=\color{blue},       % keyword style
  stringstyle=\color{mymauve},     % string literal style
}

% Bibliography

\begin{document}
% change the following line to "ngerman" for German style date and logos
%\selectlanguage{ngerman}

%title page
\begin{frame}
\titlepage
\end{frame}

%table of contents
\begin{frame}{Gliederung}
 \tableofcontents
\end{frame}

\section{Binärer Heap}
\begin{frame}[fragile]{Binärer Heap}

\hspace{13em}
\includegraphics[scale=0.4]{Binary_heap_indexing}
 \begin{block}{Definition}
  Ein Binärbaum H ist ein Min-Heap, wenn für jeden Knoten $i$ mit $i \neq \text{root}$ gilt:
  $$\text{key}(H,i) \geq \text{key}(H,\text{parent}(i)$$
  und jede Schicht, bis auf die letzte, ausgefüllt ist.
  Wobei \verb|parent(i)| den Elterknoten von $i$ zurückgibt.
 \end{block}

\end{frame}

\begin{frame}{Binary Heap -- Aufgabe}
Geben sie jeden möglichen Max Heap zu den Zahlen 1,4,3,9,6 an.
\end{frame}

\begin{frame}{Bulk Insertion bei binären Heaps}
Gegeben sei ein binärer Heap, der n Elemente enhält. Es sollen nun k Elemente auf einmal
eingefügt werden. Geben Sie ein Verfahren an (kein Pseudocode), mit dem man das Einfügen in $\Oh(\min\{k \log k + \log n, k + \log n \log k \})$ Schritten erledigen kann.
Sie können davon ausgehen, dass der Heap genau $2m - 1$ enthält ($m \in \N$).
\end{frame}

\begin{frame}{Pancake-Sorting}
 Gegeben sind n Pancakes in unterschiedlicher Größe und gestapelt. Man hat einen Pancakeflipper zur Verfügung mit dem man
 die obersten l Pancakes umdrehen kann.
 Entwickelt einen schnellen Algorithmus, um die Pancakes umzudrehen.
\end{frame}

\begin{frame}{Spaghetti Sort}
 
\end{frame}


\end{document}
