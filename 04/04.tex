%% LaTeX-Beamer template for KIT design
%% by Erik Burger, Christian Hammer
%% title picture by Klaus Krogmann
%%
%% version 2.1
%%
%% mostly compatible to KIT corporate design v2.0
%% http://intranet.kit.edu/gestaltungsrichtlinien.php
%%
%% Problems, bugs and comments to
%% burger@kit.edu

\documentclass[18pt]{beamer}
%% SLIDE FORMAT
\usepackage[utf8]{inputenc}
\usepackage{enumitem}
% use 'beamerthemekit' for standard 4:3 ratio
% for widescreen slides (16:9), use 'beamerthemekitwide'

%\usepackage{templates/beamerthemekit}
\usepackage{wrapfig}
\usepackage{hyperref}
\usepackage{algpseudocode}
 \newcommand{\real}{\mathbb{R}}
 \newcommand{\nat}{\mathbb{N}}
 \newcommand{\Oh}{\mathcal{O}}
 \newcommand{\oh}{\mathrm{o}}
 \newcommand{\SP}{\mathrm{SP}}

% \usepackage{templates/beamerthemekitwide}

%\usepackage{epigraph}

% for quotes

\AtBeginSection[] % Do nothing for \section*
{
\begin{frame}<beamer>
\frametitle{Gliederung}
\tableofcontents[currentsection]
\end{frame}
}


\usepackage{biolinum}

\usepackage{pgfplots}
\usepackage{filecontents}
\begin{filecontents}{data1.dat}
 7 1
 11.5 2
 12 1
 12.5 2
 14 2
 18 4
\end{filecontents}

\begin{filecontents}{data2.dat}
8.5 1
12.5 2
13 2
14.5 1
15.5 4
\end{filecontents}

\begin{filecontents}{data3.dat}
 22.5 2
 20.5 2
 16 2
 12.5 1
 6.5 2
\end{filecontents}


%% TikZ INTEGRATION

% use these packages for PCM symbols and UML classes
% \usepackage{templates/tikzkit}
% \usepackage{templates/tikzuml}

% the presentation starts here

\title[Algo I Tut]{4. Algorithmen Tutorium I}
\subtitle{Datenstrukturen}
\author[Zangerle]{Konstantin Zangerle}

\institute{Institut für Theoretische Informatik}

\usepackage{listings}
\usepackage{color}

\definecolor{mygreen}{rgb}{0,0.6,0}
\definecolor{mygray}{rgb}{0.5,0.5,0.5}
\definecolor{mymauve}{rgb}{0.58,0,0.82}

\lstset{ %
  backgroundcolor=\color{white},   % choose the background color
  basicstyle=\footnotesize,        % size of fonts used for the code
  breaklines=true,                 % automatic line breaking only at whitespace
  captionpos=b,                    % sets the caption-position to bottom
  commentstyle=\color{mygreen},    % comment style
  escapeinside={\%*}{*)},          % if you want to add LaTeX within your code
  keywordstyle=\color{blue},       % keyword style
  stringstyle=\color{mymauve},     % string literal style
}

% Bibliography

\begin{document}

% change the following line to "ngerman" for German style date and logos
%\selectlanguage{ngerman}

%title page
\begin{frame}
\titlepage
\end{frame}

%table of contents
\begin{frame}{Gliederung}
 \tableofcontents
\end{frame}


\section{Besprechung der Übungsaufgaben}
\begin{frame}{Übungsblatt 1}
\begin{tikzpicture}
 \begin{axis}
  \addplot plot file{data1.dat};
 \end{axis}
\end{tikzpicture} 
\end{frame}

\begin{frame}[fragile]{Übungsblatt 2}
Häufige Fehler
\begin{itemize}
 \item ?
\end{itemize}
  \begin{tikzpicture}[scale=0.5]
 \begin{axis}
  \addplot plot file{data2.dat};
 \end{axis}
\end{tikzpicture}
\end{frame}

\begin{frame}{Übungblatt 3}
Häufige Fehler
\begin{itemize}
 \item ?
\end{itemize}
  \begin{tikzpicture}[scale=0.5]
 \begin{axis}
  \addplot plot file{data3.dat};
 \end{axis}
\end{tikzpicture}
\end{frame}

\section{Wahrscheinlichkeitstheorie}
\begin{frame}{Wahrscheinlichkeitstheorie für den Hausgebrauch}
 \begin{itemize}
  \item Ergebnisraum
  \item Elementarereignisse
  \item Ereignisse
  \item $\sum_x p_x = \pause 1$
  \item Gleichverteilung
  \item Zufallsvariable
  \item 0-1-Indikator
  \item Erwartungswert \pause $E[X_0] = \sum_{y \in \Omega} p_y X(y)$
 \end{itemize}

\end{frame}

\begin{frame}{Bälle und Urnen}
 \begin{tabular}{lll}
  --- 				&	mit Zurücklegen 	& ohne Zurücklegen 	\\
 mit Beachtung d. Reihenfolge  	& 	$n^k$			& $\frac{n!}{(n-k)!}$	\\
 ohne Beachtung d .Reihenfolge 	& 	$\binom{n+k-1}{n-1}$	& $\binom{n}{k}$
 \end{tabular}

\end{frame}

\begin{frame}{Aufgaben}
 Ein volles Tutorium bestimmt aus seinen 15 Algorithmen seine Top 6. Wie viele Möglichkeiten dafür gibt es?
 \pause
 Ziehe 6 aus 15 ohne Zurücklegen, beachte die Reihenfolge
 $$ \frac{n!}{(n-k)!} = \frac{15!}{9!} = 3 603 600$$
\end{frame}

\begin{frame}{Weitere Aufgabe}
 3 interessante und 5 langweilige Module sollen auf 8 Semester aufgeteilt werden, also jedes Semester ein Modul. 
 Wie viele Möglichkeiten gibt es diese zu verteilen, wenn die interessanten und langweiligen Module untereinander 
 nicht unterscheidbar sind?
 \pause
 Ziehe zunächst für die interessanten eine Zahl, danach für die langweiligen.
 $$ \binom{8}{3} \binom{5}{5} = \frac{8!}{3! 5!} = 56$$
\end{frame}


\section{SparseArray}
\begin{frame}{SparseArray}
\begin{itemize}
 \item 	Entwerfe ein \textit{SparseArray} 
 \item 	Datenstruktur mit den Eigenschaften eines beschränkten Arrays, 
  \begin{itemize}
   \item + schnelle Erzeugung
   \item + schneller Reset
  \end{itemize}
 \item \textbf{allocate} liefert \emph{uninitialisierten} Speicher in konstanter Zeit.
	\begin{itemize}
	\item $n$ Slots braucht $\Oh(n)$ Speicher.
	\item leeren \textit{SparseArray} mit $n$ Slots braucht $\Oh(1)$ Zeit. 
	\item unterstützt eine Operation
	\begin{itemize}
	 \item \textit{reset}, in $\Oh(1)$ Zeit in leeren Zustand.
	 \item $\mathit{get}(i)$ und $\mathit{set}(i,x)$ mit wahlfreien Zugriff
	\end{itemize}
	\item $i$-te liefert zunächst $A.\mathit{get}(i)$ einen speziellen Wert $\bot$.
	\end{itemize}
\end{itemize}
 
\end{frame}

\begin{frame}{Lösung}
 \begin{itemize}
  \item Nutze zwei Arrays: D für Nutzdaten, H für Indizes.
  \item In D zusätzlich jeweils eine nicht-negative ganze Zahl
  \item Zähler, inital auf 0. \pause
  \item $\mathit{set}(i,x)$
    \begin{itemize}
     \item $D[i].\mathit{daten}:=x$, 
     \item $D[i].\mathit{zahl}:=c$ 
     \item $H[c]:=i$.
     \item Danach wird $c$ um eins erhöht.
    \end{itemize}
  \item $\mathit{get}(i)$
    \begin{itemize}
     \item Wurde $i$-te Slot gesetzt?
     \item $D[i].\mathit{zahl} < c$
     \item $H[D[i].\mathit{zahl}] = i$
    \end{itemize}
  \item \textit{reset}: Setzte $c=0$
 \end{itemize}

\end{frame}



\begin{frame}
\includegraphics[scale=0.7]{correlation} \\
License: http://creativecommons.org/licenses/by-nc/2.5/ \\
by xkcd.com/552
\end{frame}


\end{document}
